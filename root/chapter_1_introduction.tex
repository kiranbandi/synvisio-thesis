\chapter{Introduction}

With the emergence of new sequencing systems, genomic data is being generated at an unprecedented rate. Almost two decades back \textit{The Human Genome Project} took 13 years and over 3 billion dollars to sequence the entire human genome whereas the same information can be sequenced today in under an hour for 1000 dollars. This rapid improvement in sequencing has improved the availability of high-resolution genomics data and has helped researchers in tackling a wide range of biological questions.


% reference to https://www.genome.gov/human-genome-project/Completion-FAQ


An essential field in biological research where genomic data is extensively used is comparative genomics.It involves comparing genomic information between different species to understand their similarity. A genome of an organism consists of its complete set of DNA as a collection of genes where every gene is a sequence that is responsible for one or more traits in that organism. Comparing genomic sequences between two different organisms can help researchers in understating their evolutionary relationship as similar sequences can often mean that the genes have the same function. Such similar sequences are referred to as homologous sequences, and they indicate shared ancestry. As organisms evolve overtime and diversify into different species, they retain parts of their DNA from their common ancestor. The study of these conserved homologous regions is called \textbf{Synteny}. 

While a considerable part of comparing large scale genomic sequences is purely computational and thus can be automated, human judgment is still vital in synteny analysis. Visual data exploration, for example, can help researchers in easily identifying similarities among large scale genomes as humans are intuitively good at picking out patterns in pictures and visuals. Synteny visualization commonly involves visualizing genomes at the whole genome level or the individual chromosome level and representing similar genes either by connected links or the same colored regions. Syntenic data analysis can often be an iterative process where researchers visualize computational results multiple times under various parameters such as the size and orientation of similar genes based on a given biological hypothesis.

The choice of visual encoding in the representation of syntenic relationship is dependant on the kind of analysis that is being done by genome researchers. Certain graphical representations like dot plots where every conserved gene is represented as a point on a two dimensional matrix, are useful in analyzing extremely large genomes in a single representation as shown in Figure \ref{fig:ch_1_dot_plot} while other representations like parallel link plots where syntenic links are represented as coloured ribbons connecting similar regions are useful in performing a more in depth analysis as the conserved regions are more visually prominent. Additionally Circos plots which use a circular ideogram layout, as shown in Figure \ref{fig:ch_1_circos_plot}, are also frequently used by researchers in publications as they can be aesthetically pleasing to the general public while still being informative enough for researchers.

With such varied graphical representations, arriving at the right form of visualization can be difficult, and any system that offers only a single kind of visual encoding can become limited in its usability for a wide range of biological scenarios. Apart from looking at synteny in multiple representations, researchers are also often interested in  investigating specific conserved regions further and thus require the visualization system to be adaptive based on the genomic scale of interest. Thus visualizations systems need to go beyond acting as basic chart generating systems and instead offer a rich interactive experience where researchers can explore sequences from the whole genome level all the way down to the individual gene level in multiple graphical representations.

A key part of every visualization system is the data that drives it and genomic data owing to its large volume, is being increasingly managed and distribution through several online databases like NCBI and genBank.This rapid dissemination of data across the internet has also created a need for visualization systems to be easily available across the web so researchers can collaborate and share their work.


\begin{figure}
\centering
\begin{minipage}{.5\textwidth}
  \centering
  \includegraphics[width=.75\linewidth]{images/ch_1_dot_plot.PNG}
  \captionof{figure}{Dot plot}
  \label{fig:ch_1_dot_plot}
\end{minipage}%
\begin{minipage}{.5\textwidth}
  \centering
  \includegraphics[width=.75\linewidth]{images/ch_1_circos_plot.PNG}
  \captionof{figure}{Circos Plot}
  \label{fig:ch_1_circos_plot}
\end{minipage}
\end{figure}


\section{Problem and Motivation}

The problem addressed in this thesis is: \textit{existing synteny tools are limited in their accessibility, offer little or no interactive experience, and aren't integrated with the existing synteny detection tools to provide a seamless experience.}

Owing to the complexity involved in generating visualizations of large scale genomes, synteny visualization tools are primarily command-line based or stand-alone programs limited to working in specific operating systems. This combined with the steep learning curve in using these systems, means that a broad set of these tools aren't accessible to the wider science community. Of the few online visualizations systems that exist, most act as simple chart generation systems instead of offering researchers the chance to explore their datasets. This has mostly pushed visualization into the report generation stage instead of the  iterative hypothesis testing phase of the research cycle.

Understanding genomic conservation is crucial for researchers as it has applications in a wide variety of scenarios, such as predicting whole-genome duplication events; annotating extremely large genome sequences like wheat; and classifying the proximity of different species in their evolutionary history.
While visualization systems are essential as report generating tools that can create publication-ready charts, they need to move to earlier stages of the research cycle to accelerate the process of hypothesis testing.
Researchers should have the ability to interact with their datasets and change parameters in real-time to see their results in easily understandable visuals,
which in turn, can let researchers explore a wide range of biological scenarios in a shorter period of time.
% refer vgsc

\section{Solution}

\begin{figure}
  \centering
  \includegraphics[width=.75\linewidth]{images/ch_1_dashboard.PNG}
  \captionof{figure}{Syteny Dashboard visualizing genome collinearity in Bn(Brassica Napus) with the following components: \textbf{a)} Linear Link Plot with connected ribbons representing collinear gene blocks. \textbf{b)}Dot plot where every collinear gene is represented by a point and contiguous collinear blocks are shown as lines. \textbf{c)} Filter panel representing all the collinear blocks based on the count of their genes with ability to refine results using slider to the left.  }
  \label{fig:ch_1_dashboard}
\end{figure}

To address the lack of proper analysis tools in syteny research, we developed \textbf{SynVisio}, a decentralized web-based environment for browsing syntenic blocks with multiple visual representations across different scales from the whole genome level to the individual gene level.

\textbf{SynVisio} can directly work with the results of existing synteny detecting tools like MCScanX\cite{wang2012mcscanx} and DAGChainer\cite{haas2004dagchainer} and can visualize conservation in multiple representations. It works in two modes,the primary synteny analysis mode lets users compare chromosomes in the same genome or between two genomes, and the information is visualized as Parallel link plots, Dot plots, or both as shown in Figure \ref{fig:ch_1_dashboard}. For visualizing synteny across several genomes simultaneously \textbf{SynVisio} offers a Multi-level analysis mode where synteny is visualized in stacked Parallel plots or Hive plots.\textbf{SynVisio} offers a rich interactive experience by letting users switch graphs in real-time and explore data from the genome level all the way down to the individual gene level. Users can do this by simply clicking on any two chromosomes when looking at the visualization in the genome level and then further step down from the individual chromosome level by clicking on a particular gene block to look at its constituent genes and their orientation. Additionally, users can also annotate their charts with additional genomic data in the form of tracks above the genomes or chromosomes, which can be visualized as heat-maps, histograms or scatter-plots.



By default, \textbf{SynVisio} operates as a dashboard and lets users select the chromosomes that they wish to explore. It then visualizes the conserved gene blocks through coordinated multiple views consisting of a Parallel plot, Dot plot and an adaptive filter panel where users can refine the conserved regions using a slider based on the level of similarity and the number of contiguous genes in a conserved block.
As users explore synteny using \textbf{SynVisio} it offers them the ability to records their interactions as snapshots, which can be revisited or reset. This gives users the ability to examine multiple scenarios and switch between them. The system also indexes all the conserved genes in the browser, thus letting users quickly lookup genes by their \textit{gene IDs} to see which conserved blocks they belong to. Finally, \textbf{SynVisio} offers users the ability to download all the charts in transform and scale invariant vector graphics for research publication.

\section{Steps to the Solution} 
There were several steps involved in designing a system that could let researchers explore syteny through an easily accessible web based based tool.

\begin{itemize}
    
    \item \textbf{Formulate Design Requirements} -
    To characterize the needs from the biological research community, we primarily met with three groups of researchers studying genomic conservation through a series of structured interviews to collect their requirements. The first group we met was interested in exploring synteny in wheat while the other two groups were involved in studying Canola and Pulse crops respectively. All three groups were unanimous in the verdict that synteny is a critical issue to study for understanding genomic evolution and that existing tools don't meet their needs.Based on the feedback from the genomic research community all requirements can be broadly classified into either \textbf{functional} or \textbf{non functional} requirements.Functional requirements primarily include understanding the size, location, and orientation of conserved sequences along with having the ability to filter sequences based on their similarity while non-functional requirements include features like the ability to download images or snapshot explorational points to revisit.
    
    \item \textbf{Research Existing Alternatives} - 
    Since synteny analysis is a combination of synteny detection followed by downstream analysis using visualization systems, we looked at tools that operate in both these domains. We looked at several state of the art syteny detection packages like MCScanX, DAGChainer, Cyntenator and i-ADHoRe \cite{wang2012mcscanx,haas2004dagchainer,rodelsperger2010cyntenator,proost2011adhore}. We also tested the visual outputs of some that had their own downstream analysis tools.We focused our research on MCScanX and DAGChainer out of the other alternatives as they were the most popular and frequently used tools by the researchers we interviewed and had more accessible and  efficient output data formats in the form of syntenic blocks or orthologue tables. We followed this by looking at the tools that worked in the second stage of the analysis pipeline by providing visualizations, like SynChro, GSV, Mizbee, VGSC, and Circos\cite{drillon2014synchro,revanna2011gsv,Meyer2009,xu2016vgsc}. Of these, we found that most served as simple graphic generating systems instead of offering a platform for detailed analysis except for MizBee, which was however, limited by its accessibility and small variety of charts. We finally refined our initial requirements based on the characterization of the problem domain by some of these earlier tools like MizBee~\cite{Meyer2009}.

    \item \textbf{Choice of Visual Encoding and System Architecture} - 
    To implement our solution, we decided on a web-based single page tool that would work as a part of the existing analysis pipeline by working directly on the results of existing synteny detection tools. We adopted a thick-client architecture model as opposed to a thin client model where visualizations are generated on the server as this would let researchers directly upload their synteny analysis files and see the resulting images in real time without their sensitive data being sent to a remote server. To visually represent genomic conservation, we used linear connections (parallel plots)and points(dot plots). We then encoded additional information about the size and orientation of the gene blocks through a combination of colour and shape.
    To render the visualizations, we used both canvas and simple vector graphics and tested both and found that the latter while being more resource intensive offered a better visual experience across multiple device sizes and resolutions as it was transform and scale variant. So our system was designed to render all charts as simple vector graphics initially but can dynamically switch to canvas rendering for large scale genomes. The final application was developed through several design iterations as the system was used by our expert user group and several additional features not in our original requirements like support for additional tracks and the ability to download images were added based on their feedback. 

\end{itemize}

\section{Evaluation}

We deployed a stable version of our visualization system online free for public use and collected user logs through google analytics over a period of 6 months to quantify user engagement with our system. Our analysis of the logs revealed that our system was found to helpful by a large group of users in the international research community spread across seven major countries.

-Need to explore this section more based on the logs which country how many people used it when used etc.

Due to the high level of expertise involved in using our tool we did not run a a user study on regular participants and instead opted for structured interviews with a small group of genome researchers.

\section{Thesis Outline}

Final thesis outline section to be filled in once all other sections are complete.