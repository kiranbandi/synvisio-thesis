\chapter{Conclusion}

Comparative genomic research plays a vital role in studying genome evolution and ancestral genome reconstruction. However, despite the availability of high-resolution genomic data, research in this field is being restricted due to the lack of proper analysis tools. While some analysis tasks can be automated to deal with the high volume of data, other tasks still require manual interpretation such as synteny analysis. Visualizing data in such scenarios can help researchers in their analysis by offloading part of the cognitive load required in processing information onto humans' inherent capacity for visual perception. Visualizing synteny blocks can aid researchers in understanding the location, size, and orientation of conserved genomic regions. Although some tools do exist for synteny analysis, they are limited in their usability and offer very little interactivity needed to explore complex datasets. Our primary contribution in this research work is SynVisio, a synteny analysis tool that offers genomic researchers different ways to visualize and explore genomic data. Researchers can access the tool through a public web-based interface and directly upload their synteny analysis files. Information can be analyzed in a primary analysis mode through pairwise comparative visualizations such as linear parallel plots and dot plots. Alternatively, researchers also have access to a multi genome analysis mode where syntenic blocks can be visualized through hive plots or stacked parallel plots to trace genomic conservation across several genomes at once.

Our second contribution was in adding interactive support to our system to help researchers in refining and enhancing their datasets. All visualizations are accompanied by a filter panel to modify the generated visualizations in real time. Syntenic blocks can be filtered based on the level of similarity (score or number of genes in a block) or the probability of the match (E value) depending on the underlying genomic question. Researchers can also augment certain visualizations such as parallel plots and dot plots with tracks representing additional information such as gene density or SNP count. These tracks are in the form of heat-maps, line charts, scatter plots or histograms. The tracks along with all visualizations can also be exported in publication-ready formats.

Our third contribution is providing support for revisitation. Synteny analysis is an exploratory task that requires researchers to investigate conservation at multiple genomic resolutions. Such exploratory analysis requires users to switch between multiple visualizations under different filter parameters. This switching can, however, cause them to lose context of their position in the dataset. SynVisio avoids this by providing users the option to snapshot the state of the system at any point in their exploratory analysis for easy and quick revisitation. This along with other features such as searching for genes in syntenic blocks, can be useful to researchers, particularly in exploring large datasets.

SynVisio has been developed as a modular component that can be reused in existing online genomic analysis tools, and the source code for the system has been open-sourced to facilitate the rapid dissemination of our work into other scenarios. Several researchers are currently using our system across the world either directly via the web interface or through the integration of our system into their existing tools. We also plan on adding additional features to the system in the coming year to offer support for other kinds of genomic analysis tasks.
