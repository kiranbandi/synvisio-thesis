\documentclass{uofsthesis-cs}

% Documentation for the uofsthesis-cs class is given in uofsthesis-cs.dvi
% 
% It is recommended that you read the CGSR thesis preparation
% guidelines before proceeding.
% They can be found at http://www.usask.ca/cgsr/thesis/index.htm

%%%%%%%%%%%%%%%%%%%%%%%%%%%%%%%%%%%%%%%%%%%%%%%%%%%%%%%%%%%%%%%%%%%%%%%%%%%%%%
% FRONTMATTER - In this section, specify information to be used to
% typeset the thesis frontmatter.
  \usepackage{graphicx} 
\usepackage{cite}   
\usepackage{url}  
\usepackage{todonotes}
\usepackage{textcomp}
%%%%%%%%%%%%%%%%%%%%%%%%%%%%%%%%%%%%%%%%%%%%%%%%%%%%%%%%%%%%%%%%%%%%%%%%%%%%

% THESIS TITLE
% Specify the title. Set the capitalization how you want it.
\title{SynVisio: A Multiscale Tool for Visualizing Genomic Conservation}

% AUTHOR'S NAME
% Your name goes here.
\author{Venkat Kiran Bandi}

% DEGREE SOUGHT.  
% Use \MSc or \PhD here
\degree{\MSc}         

% THESIS DEFENCE DATE
% Should be month/year, e.g. July 2004
\defencedate{December/2019}


% NAME OF ACADEMIC UNIT
%
% The following two commands allow you to specify the academic unit you belong to.
% This will appear on the title page as
% ``<academic unit> of <department>''.
% So if you are in the division of biomedical engineering you would need to do:
% \department{Biomedical Engineering}
% \academicunit{Division}
%
% The default is ``Department of Computer Science'' if these commands
% are not given.
%
% If you are in a discipline other than Computer Science, uncomment the following line and
% specify your discipline/department.  Default is 'Computer Science'.
% \department{If not Computer Science, put the name of your department here}

% If you are not in a department, but say, a division, uncomment the following line.
% \academicunit{Put the type of academic unit you belong to here, e.g. Division, College}


% PERMISSION TO USE ADDRESS
%
% If you are not in Comptuer Science you will want to change the
% address on the Permission to Use page.  This is done using the
% \ptuaddress{}.  Example:
%
% \ptuaddress{Head of the Department of Computer Science\\
% 176 Thorvaldson Building\\
% 110 Science Place\\
% University of Saskatchewan\\
% Saskatoon, Saskatchewan\\
% Canada\\
% S7N 5C9
% }

% ABSTRACT
\abstract{
Due to rapid advancements in sequencing technologies high resolution genomic data is readily available for a wide range of species. However, analysing this huge volume of data is still a tedious task. While most genomic analysis tasks can be automated some still require human judgement and manual interpretation. One example of such a task is synteny analysis. Synteny is a key tool in comparative genomic research - it is the study of homologous regions within chromosomes of the same or different species. Visualizing synteny can help researchers in understanding the \textbf{location, size and orientation} of shared sequences among genomes of interest. However, the current tools which exist for synteny analysis are stand-alone programs or command line tools that are difficult to use and that offer very little interactivity. In this paper we provide a decentralized web-based environment for browsing syntenic blocks with multiple visual representations in the form of linear charts, hive charts and dot plots. Our tool also makes it easy to save and revisit graphs so that researchers can iteratively refine their plots and compare them with previous revisions for changes. We present SynVisio, a multiscale synteny analysis tool that provides multiple visualizations of genome collinearity for easy exploration across different scales from the whole genome level all the way down to the individual gene block level.
}

% THESIS ACKNOWLEDGEMENTS -- This can be free-form.
\acknowledgements{
Acknowledgements go here.  Typically you would at least thank your supervisor.
}

% THESIS DEDICATION -- Also free-form.  If you don't want a dedication, comment out the following
% line.
% \dedication{This is the thesis dedication (optional)}

% LIST OF ABBREVIATIONS - Sample  
% If you don't want a list of abbreviations, comment the following 4 lines.
\loa{
\abbrev{DNA}{Deoxyribonucleic Acid}
\abbrev{FASTA}{Fast All}
\abbrev{SVG}{Simple Vector Graphics}
}

%%%%%%%%%%%%%%%%%%%%%%%%%%%%%%%%%%%%%%%%%%%%%%%%%%%%%%%%%%%%%%%%
% END OF FRONTMATTER SECTION
%%%%%%%%%%%%%%%%%%%%%%%%%%%%%%%%%%%%%%%%%%%%%%%%%%%%%%%%%%%%%%%%

\begin{document}

% Typeset the title page
\maketitle

% Typeset the frontmatter.  
\frontmatter

%%%%%%%%%%%%%%%%%%%%%%%%%%%%%%%%%%%%%%%%%%%%%%%%%%%%%%%%%%%%%%%%
% FIRST CHAPTER OF THESIS BEGINS HERE
%%%%%%%%%%%%%%%%%%%%%%%%%%%%%%%%%%%%%%%%%%%%%%%%%%%%%%%%%%%%%%%%
\chapter{Introduction}

With the emergence of new sequencing systems, genomic data is being generated at an unprecedented rate. Almost two decades back \textit{The Human Genome Project} took 13 years and over 3 billion dollars to sequence the entire human genome whereas the same information can be sequenced today in under an hour for 1000 dollars. This rapid improvement in sequencing has improved the availability of high-resolution genomics data and has helped researchers in tackling a wide range of biological questions.


% reference to https://www.genome.gov/human-genome-project/Completion-FAQ


An essential field in biological research where genomic data is extensively used is comparative genomics.It involves comparing genomic information between different species to understand their similarity. A genome of an organism consists of its complete set of DNA as a collection of genes where every gene is a sequence that is responsible for one or more traits in that organism. Comparing genomic sequences between two different organisms can help researchers in understating their evolutionary relationship as similar sequences can often mean that the genes have the same function. Such similar sequences are referred to as homologous sequences, and they indicate shared ancestry. As organisms evolve overtime and diversify into different species, they retain parts of their DNA from their common ancestor. The study of these conserved homologous regions is called \textbf{Synteny}. 

While a considerable part of comparing large scale genomic sequences is purely computational and thus can be automated, human judgment is still vital in synteny analysis. Visual data exploration, for example, can help researchers in easily identifying similarities among large scale genomes as humans are intuitively good at picking out patterns in pictures and visuals. Synteny visualization commonly involves visualizing genomes at the whole genome level or the individual chromosome level and representing similar genes either by connected links or the same colored regions. Syntenic data analysis can often be an iterative process where researchers visualize computational results multiple times under various parameters such as the size and orientation of similar genes based on a given biological hypothesis.

The choice of visual encoding in the representation of syntenic relationship is dependant on the kind of analysis that is being done by genome researchers. Certain graphical representations like dot plots where every conserved gene is represented as a point on a two dimensional matrix, are useful in analyzing extremely large genomes in a single representation as shown in Figure \ref{fig:ch_1_dot_plot} while other representations like parallel link plots where syntenic links are represented as coloured ribbons connecting similar regions are useful in performing a more in depth analysis as the conserved regions are more visually prominent. Additionally Circos plots which use a circular ideogram layout, as shown in Figure \ref{fig:ch_1_circos_plot}, are also frequently used by researchers in publications as they can be aesthetically pleasing to the general public while still being informative enough for researchers.

With such varied graphical representations, arriving at the right form of visualization can be difficult, and any system that offers only a single kind of visual encoding can become limited in its usability for a wide range of biological scenarios. Apart from looking at synteny in multiple representations, researchers are also often interested in  investigating specific conserved regions further and thus require the visualization system to be adaptive based on the genomic scale of interest. Thus visualizations systems need to go beyond acting as basic chart generating systems and instead offer a rich interactive experience where researchers can explore sequences from the whole genome level all the way down to the individual gene level in multiple graphical representations.

A key part of every visualization system is the data that drives it and genomic data owing to its large volume, is being increasingly managed and distribution through several online databases like NCBI and genBank.This rapid dissemination of data across the internet has also created a need for visualization systems to be easily available across the web so researchers can collaborate and share their work.


\begin{figure}
\centering
\begin{minipage}{.5\textwidth}
  \centering
  \includegraphics[width=.75\linewidth]{images/ch_1_dot_plot.PNG}
  \captionof{figure}{Dot plot}
  \label{fig:ch_1_dot_plot}
\end{minipage}%
\begin{minipage}{.5\textwidth}
  \centering
  \includegraphics[width=.75\linewidth]{images/ch_1_circos_plot.PNG}
  \captionof{figure}{Circos Plot}
  \label{fig:ch_1_circos_plot}
\end{minipage}
\end{figure}


\section{Problem and Motivation}

The problem addressed in this thesis is: \textit{existing synteny tools are limited in their accessibility, offer little or no interactive experience, and aren't integrated with the existing synteny detection tools to provide a seamless experience.}

Owing to the complexity involved in generating visualizations of large scale genomes, synteny visualization tools are primarily command-line based or stand-alone programs limited to working in specific operating systems. This combined with the steep learning curve in using these systems, means that a broad set of these tools aren't accessible to the wider science community. Of the few online visualizations systems that exist, most act as simple chart generation systems instead of offering researchers the chance to explore their datasets. This has mostly pushed visualization into the report generation stage instead of the  iterative hypothesis testing phase of the research cycle.

Understanding genomic conservation is crucial for researchers as it has applications in a wide variety of scenarios, such as predicting whole-genome duplication events; annotating extremely large genome sequences like wheat; and classifying the proximity of different species in their evolutionary history.
While visualization systems are essential as report generating tools that can create publication-ready charts, they need to move to earlier stages of the research cycle to accelerate the process of hypothesis testing.
Researchers should have the ability to interact with their datasets and change parameters in real-time to see their results in easily understandable visuals,
which in turn, can let researchers explore a wide range of biological scenarios in a shorter period of time.
% refer vgsc

\section{Solution}

\begin{figure}
  \centering
  \includegraphics[width=.75\linewidth]{images/ch_1_dashboard.PNG}
  \captionof{figure}{Syteny Dashboard visualizing genome collinearity in Bn(Brassica Napus) with the following components: \textbf{a)} Linear Link Plot with connected ribbons representing collinear gene blocks. \textbf{b)}Dot plot where every collinear gene is represented by a point and contiguous collinear blocks are shown as lines. \textbf{c)} Filter panel representing all the collinear blocks based on the count of their genes with ability to refine results using slider to the left.  }
  \label{fig:ch_1_dashboard}
\end{figure}

To address the lack of proper analysis tools in syteny research, we developed \textbf{SynVisio}, a decentralized web-based environment for browsing syntenic blocks with multiple visual representations across different scales from the whole genome level to the individual gene level.

\textbf{SynVisio} can directly work with the results of existing synteny detecting tools like MCScanX\cite{wang2012mcscanx} and DAGChainer\cite{haas2004dagchainer} and can visualize conservation in multiple representations. It works in two modes,the primary synteny analysis mode lets users compare chromosomes in the same genome or between two genomes, and the information is visualized as Parallel link plots, Dot plots, or both as shown in Figure \ref{fig:ch_1_dashboard}. For visualizing synteny across several genomes simultaneously \textbf{SynVisio} offers a Multi-level analysis mode where synteny is visualized in stacked Parallel plots or Hive plots.\textbf{SynVisio} offers a rich interactive experience by letting users switch graphs in real-time and explore data from the genome level all the way down to the individual gene level. Users can do this by simply clicking on any two chromosomes when looking at the visualization in the genome level and then further step down from the individual chromosome level by clicking on a particular gene block to look at its constituent genes and their orientation. Additionally, users can also annotate their charts with additional genomic data in the form of tracks above the genomes or chromosomes, which can be visualized as heat-maps, histograms or scatter-plots.



By default, \textbf{SynVisio} operates as a dashboard and lets users select the chromosomes that they wish to explore. It then visualizes the conserved gene blocks through coordinated multiple views consisting of a Parallel plot, Dot plot and an adaptive filter panel where users can refine the conserved regions using a slider based on the level of similarity and the number of contiguous genes in a conserved block.
As users explore synteny using \textbf{SynVisio} it offers them the ability to records their interactions as snapshots, which can be revisited or reset. This gives users the ability to examine multiple scenarios and switch between them. The system also indexes all the conserved genes in the browser, thus letting users quickly lookup genes by their \textit{gene IDs} to see which conserved blocks they belong to. Finally, \textbf{SynVisio} offers users the ability to download all the charts in transform and scale invariant vector graphics for research publication.

\section{Steps to the Solution} 
There were several steps involved in designing a system that could let researchers explore syteny through an easily accessible web based based tool.

\begin{itemize}
    
    \item \textbf{Formulate Design Requirements} -
    To characterize the needs from the biological research community, we primarily met with three groups of researchers studying genomic conservation through a series of structured interviews to collect their requirements. The first group we met was interested in exploring synteny in wheat while the other two groups were involved in studying Canola and Pulse crops respectively. All three groups were unanimous in the verdict that synteny is a critical issue to study for understanding genomic evolution and that existing tools don't meet their needs.Based on the feedback from the genomic research community all requirements can be broadly classified into either \textbf{functional} or \textbf{non functional} requirements.Functional requirements primarily include understanding the size, location, and orientation of conserved sequences along with having the ability to filter sequences based on their similarity while non-functional requirements include features like the ability to download images or snapshot explorational points to revisit.
    
    \item \textbf{Research Existing Alternatives} - 
    Since synteny analysis is a combination of synteny detection followed by downstream analysis using visualization systems, we looked at tools that operate in both these domains. We looked at several state of the art syteny detection packages like MCScanX, DAGChainer, Cyntenator and i-ADHoRe \cite{wang2012mcscanx,haas2004dagchainer,rodelsperger2010cyntenator,proost2011adhore}. We also tested the visual outputs of some that had their own downstream analysis tools.We focused our research on MCScanX and DAGChainer out of the other alternatives as they were the most popular and frequently used tools by the researchers we interviewed and had more accessible and  efficient output data formats in the form of syntenic blocks or orthologue tables. We followed this by looking at the tools that worked in the second stage of the analysis pipeline by providing visualizations, like SynChro, GSV, Mizbee, VGSC, and Circos\cite{drillon2014synchro,revanna2011gsv,Meyer2009,xu2016vgsc}. Of these, we found that most served as simple graphic generating systems instead of offering a platform for detailed analysis except for MizBee, which was however, limited by its accessibility and small variety of charts. We finally refined our initial requirements based on the characterization of the problem domain by some of these earlier tools like MizBee~\cite{Meyer2009}.

    \item \textbf{Choice of Visual Encoding and System Architecture} - 
    To implement our solution, we decided on a web-based single page tool that would work as a part of the existing analysis pipeline by working directly on the results of existing synteny detection tools. We adopted a thick-client architecture model as opposed to a thin client model where visualizations are generated on the server as this would let researchers directly upload their synteny analysis files and see the resulting images in real time without their sensitive data being sent to a remote server. To visually represent genomic conservation, we used linear connections (parallel plots)and points(dot plots). We then encoded additional information about the size and orientation of the gene blocks through a combination of colour and shape.
    To render the visualizations, we used both canvas and simple vector graphics and tested both and found that the latter while being more resource intensive offered a better visual experience across multiple device sizes and resolutions as it was transform and scale variant. So our system was designed to render all charts as simple vector graphics initially but can dynamically switch to canvas rendering for large scale genomes. The final application was developed through several design iterations as the system was used by our expert user group and several additional features not in our original requirements like support for additional tracks and the ability to download images were added based on their feedback. 

\end{itemize}

\section{Evaluation}

We deployed a stable version of our visualization system online free for public use and collected user logs through google analytics over a period of 6 months to quantify user engagement with our system. Our analysis of the logs revealed that our system was found to helpful by a large group of users in the international research community spread across seven major countries.

-Need to explore this section more based on the logs which country how many people used it when used etc.

Due to the high level of expertise involved in using our tool we did not run a a user study on regular participants and instead opted for structured interviews with a small group of genome researchers.

\section{Thesis Outline}

Final thesis outline section to be filled in once all other sections are complete.
\chapter{Related Work}

\section{Genomic Conservation and Synteny Detection}
In this section we discuss the biological background behind genomic conservation and how analysing it can provide crucial answers for researchers. We also explore the framework of synteny detection and some existing tools that are currently used in genomic research to perform the same.

\subsection{Biological Background}

Genomics is the field of biology that involves the study of genomes of various organism to understand their structure, function and evolution.\cite{world2002genomics}. A genome is defined as the complete set of DNA of an organism where DNA (Deoxyribonnucleic acid) is the chemical compound containing a series of instructions responsible for the development and functioning of that organism\cite{genomegov}.All living organisms transfer genomic information from one generation to the other through through chromosomes present in the nucleus of the cell. Humans for example, have 23 pairs of chromosomes where one from every pair is inherited from each parent and is responsible for their unique traits and characteristics. A chromosome structurally is a tightly packed length of DNA along with proteins that regulate its structure and activity. DNA is made of two long strings of nucleotide bases that are wrapped around each other in a double helix structure. There are four such bases: adenine (A), guanine(G), cytosine(C) and Thymine (T) with specific rules of pairing between them such that adenine always pairs with thymine and cytosine always pairs with guanine. These nitrogenous base pairs collectively make up the entire genome of an organism. The human genome for instance, is made up of around three billion base pairs encoding information for around 20,000-25000 genes\cite{international2004finishing}. 

Genes are long segments of DNA that encode information for a specific protein which are the basic building blocks of all organisms. Proteins are made up of long chain of amino acids where the structure and function of the proteins are determinant on the order of these amino acids. Fundamentally proteins are manufactured using the information encoded in a gene through the process of transcription and translation and collectively this process is called gene expression. During transcription the DNA present in a gene acts a template to form an mRNA which is a single stranded structure consisting of one of every complementary base pair in the DNA. This is followed by translation where mRNA is used as a template to assemble a chain of amino acids such that each group of three bases in the mRNA called a codon creates one particular amino acid. Thus the order of bases in the DNA encodes for the order of amino acids in the protein and in turns its structure and function\cite{clancy2008translation}.

DNA is transferred from one generation to the other in all livings organisms through the process of self replication where the double helical structure of the DNA comes apart and each of the complementary strands acts as a template in the production of its counterpart \cite{pray2008semi} forming new pairs of DNA strands. Although cellular error checking mechanisms ensure that these new DNA strands are nearly identical to the original strand, occasionally mutations can occur. This can happen when a base at one position is replaced by one of the other bases or is entirely entirely lost alternatively insertions or duplications of longer sets of base pairs can also happen. Other kinds of chromosomal rearrangements that occur include inversions, where a large segment of chromosome is reversed in orientation or translocation where parts of chromosomes swap places between themselves\cite{hartwell2008genetics}. While most mutations that occur during duplication do not have an affect on a gene they can however in some cases alter the function of the gene. This can be detrimental in some cases leading to diseases such as cancer.While in some other cases such mutations can be beneficial by offering resistance to diseases or other environmental factors. 

\subsection{Comparative Genomics}
As these mutations accumulate over time they lead to the divergence of species. Understanding how these changes could have occurred is a major field of research in comparative genomics. Comparative genomics as the name suggests, involves comparing genome sequences of different species to identify regions of similarity and differences to gain information on the relatedness between the species genomically and functionally.

\subsection{Genome Sequencing and Assembly}

\subsection{Sequence Alignment}

\subsection{Synteny}

\subsection{Synteny Detection Tools}
DAGChainer
McScanX



\section{Genomic Visualizations} 

use cydneys paper for reference
\subsection{General visualizations} - JBrowse,GBrowse,Ensemble
\subsection{Synteny Vis systems} - VGSC,MizBee, Sybill

\section{Interactivity in Visualizations}
\subsection{Zoom in to explore, search , visualization information seeking mantra}
\subsection{Multiple linked views}
\subsection{Snapshot visual state - revisitation support}
\subsection{Annotating Charts with Tracks}


\chapter{Data and Task Abstraction}

SynVisio was developed over multiple iterations as the necessary data and requirements were refined through continuous consultation with our genome research collaborators. To understand the design choices for visual encoding and interaction that we adopted in developing SynVisio, we need to first elaborate on the underlying data and task abstractions. We start with the data abstraction, where we explain the different characteristics of the syntenic data and how it is computed using synteny detection tools and further processed by our system. We then explore the different analysis tasks that can be performed on syntenic data, and finally, we discuss supplementary requirements that help researchers in using and interacting with SynVisio.

\section{Data}
This section starts with the description of the structure of a genome and its constituent elements and follows through with an explanation of how syntenic data is generated and represented.

\subsection{Genome Structure and Scales}
The genome of every organism is unique and can be defined as the complete set of DNA consisting of a series of instructions needed to build and maintain that organism. Structurally genomes are broken down into smaller sections called chromosomes, where every chromosome is a long strand of DNA coiled up along with various proteins. Each chromosome is made up of several genes, which are the basic functional units of heredity and code for a specific protein. Genes can then be further broken down into nucleotides which are the smallest building blocks of DNA. For analyzing genomic conservation, researchers also look at a collection of collinear genes that are called blocks. Thus the genome structure can be ordered into the following five levels from top to bottom in terms of genomic size : \textit{Genome\textrightarrow{Chromosome}\textrightarrow{Block} \textrightarrow{Gene}\textrightarrow{Nucleotide}}. However, to analyze large scale genomic conservation we only look at the first four levels. The structural data of a genome describing its constituent entities is provided to us in the form of a \textit{GFF(General Feature Format) File}. It contains the start and end position of every gene on a linear scale in a chromosome, the gene identifier, and the reference name of the parent chromosome in a three tab-delimited format. A sample GFF file can be seen in figure \ref{fig:ch_3_gff_file}.  Information on several genomes belonging to different species can be present in the same file and is distinguished by the two-character key present in the reference name of the chromosomes. The data on this file is processed by SynVisio to get the genomic size (number of nucleotides) of the different genomes and their constituent chromosomes along with the size and position of all the genes within these chromosomes. This data collectively gives us a precise structural map of every genome and its different sub-elements and so can be used to visualize a genome over multiple scales/levels.

\begin{figure}[h]
  \centering
  \includegraphics[width=.55\linewidth]{images/ch_3_gff_file.PNG}
  \captionof{figure}{Sample GFF File Describing Structure of a Genome.}
  \label{fig:ch_3_gff_file}
\end{figure}


\begin{figure}[h]
  \centering
  \includegraphics[width=.675\linewidth]{images/ch_3_coll_file.PNG}
  \captionof{figure}{Sample Collinearity File with a single block highlighted.}
  \label{fig:ch_3_coll_file}
\end{figure}

\begin{figure}[h]
  \centering
  \includegraphics[width=.45\linewidth]{images/ch_3_track_file.PNG}
  \captionof{figure}{Sample Track File.}
  \label{fig:ch_3_track_file}
\end{figure}


\subsection{Conservation Data}
At the smallest level, conservation between two genes can be inferred by looking at sequence homology, which is the similarity between the nucleotide sequences. Larger genomic conservation events can be studied by looking at blocks of such homologous genes and grouping them together based on their chromosomal positions to identify collinearity. To arrive at this data, we first identify all the homologous genes between two genomes using a local alignment tool such as BLAST (Basic Local Alignment Search Tool)\cite{blasttool}. Different synteny detection tools then construct collinear blocks of these homologous genes by either clustering neighbour matching gene pairs (DAGChainer) \cite{haas2004dagchainer} or by constructing chains around an anchor gene (MCScanX)\cite{wang2012mcscanx}. These collinear blocks also referred to as syntenic blocks, are the primary source of data for our tool and are provided as a collinearity file. A sample collinearity file can be seen in figure \ref{fig:ch_3_coll_file}. Every block of collinear genes has a corresponding similarity score indicating the level of match; an expect value (\textit{e-value}) indicating the probability that the match may have been due to chance; the count of genes; the names of the source and target chromosomes the blocks of genes belong to; and the orientation of the block(forward or reverse). Finally, every block of data also consists of a list of the homologous gene pairs in that block and their statistical significance(\textit{e-value}). This data, combined with the information on the structure of the genome can be used to associate every region of a genome with its homologous regions in the other genome or within itself depending on the type of synteny under investigation.

\subsection{Auxiliary Track Data}
Since genomic data is represented as a collection of linear sequences, additional information can be provided in the form of tracks parallel to the original gene sequence structure. These tracks can contain information about genomic features such as gene density, copy number variations (CNVs), and single-nucleotide polymorphisms (SNPs). The data is provided to SynVisio in a BedGraph file format consisting of 4 tab-delimited values: chromosome identifier, chromosomal start position, chromosomal end position, and a data value. This information can be hierarchically grouped in the same way as genomic sequences into Block Level or Chromosomal Levels and can be used to annotate the corresponding sequence structure. 


\section{Tasks}

In this section, we look at the different questions researchers have when analyzing genomic conservation. We then collate all the design requirements into a series of analysis tasks and group them based on the conservation relationship they address. Finally, we also elaborate on some additional elements that improve the usability of the system and assist users in performing their analysis tasks.

\subsection{Requirement Gathering Phase}
To formulate our design requirements, we met with three groups of researchers working in distinct fields of biological research. The meetings were conducted over several iterations to refine our initial requirements. The sessions broadly revolved around understanding the basic tasks researchers perform when analyzing genomic conservation and looking at the shortcomings of the existing synteny analysis tools in the market. Although all three groups were primarily interested in analyzing synteny, their individual use cases varied, providing us with a diverse set of user scenarios.

Our first research group was involved in researching genomic conservation in the \textit{brassica} genus as it offers an ideal model to study polyploid evolution which is responsible for genetic variations that are advantageous from an evolutionary perspective\cite{madlung2013polyploidy,liu2014brassica}. In particular they were interested in understanding genomic conservation within an allotetraploid species \textit{Brassica napus}(AACC) an important oilseed crop and also in comparing it to its closely related diploid species \textit{Brassica rapa}(AA) and \textit{Brassica oleracea}(CC) that belong together in the classical triangle of U\cite{nagaharu1935genome}. The requirements from this group were focused around having access to a system that could let them visualize the conserved relationship between the different diploid species and also within the chromosomes of a single allotetraploid (self synteny). 

The second research team we met were interested in looking at genomic conservation between \textit{Lens culinaris} and \textit{Cicer arietinum} to improve various agronomic traits as these are both widely grown legume crops. The requirements here were largely focused on cross synteny rather than self synteny. A unique trait of this particular dataset is that while \textit{C. arietinum} has a genome size of ~740 Mbp, \textit{L. culinaris} has a genome size of ~4 Gbp. This vast difference in the sizes of the two genomes makes visualizing synteny at the whole genome level extremely difficult and researchers must hence rely on comparing individual chromosomes of \textit{L. culinaris} one at a time or rely on a visualization that can have variable scales between the source and the target genomes.


The final research group we met with worked in sequencing wheat and were interested in understanding the genomic conservation between the three subgenomes of the hexaploid bread wheat genome. Researchers from this team wanted a system that could visualize synteny between 3 different genomes instead of a single source and target while taking into consideration the extremely large size of the wheat genome. They were also interested in adopting a novel network-based visualization called Hive Chart, which maps nodes on a radially distributed linear axes, for exploring multi-way genomic conservation.

\subsection{Visual Analysis Tasks}
Based on the underlying data and the requirements gathered from our research collaborators, we can group all analysis tasks performed in exploring genomic conservation into three basic groups according to the genomic scale at which they operate.

\smallskip
\noindent
\textbf{Genome Level}
\smallskip
\begin{enumerate}
\item [$Q_1$.] What is the level of conservation that exists between two or more sets of genomes?
\item [$Q_2$.] How does the density of conservation change across the genomes, and are there any gaps?
\item [$Q_3$.] How does the ordering of chromosomes based on conservation change between a given set of genomes or within a single genome? (possibility of detecting whole-genome duplication or genome reversal )
\item [$Q_4$.] If unmarked scaffolds exist, which regions of the target genome do they share similarity with?
\item[$Q_5$] Which chromosomes are sparsely or entirely unaligned, and how does the level of conservation change when these are ignored ?
\end{enumerate}

\smallskip
\noindent
\textbf{Chromosome Level}
\smallskip
\begin{enumerate}
\item [$Q_6$.]What is the level of conservation between a specific subset of chromosomes?
\item [$Q_7$.] What is the level of conservation between a single chromosome and an entire target genome or several other chromosomes (detecting unaligned regions within a chromosome)?
\item [$Q_8$.] 
How large are the collinear blocks relative to neighbouring chromosomes?
\item [$Q_9$.]What is the orientation of collinear blocks between two given chromosomes? (regular or inverted)
\end{enumerate}

\smallskip
\noindent
\textbf{Block Level}
\smallskip
\begin{enumerate}
\item [$Q_{10}$.]What is the level of conservation between the set of genes in a collinear block?
\item [$Q_{11}$.] What are the different genes contained in a block?
\item [$Q_{12}$.] What is the size of a gene relative to the size of the collinear block?
\item [$Q_{13}$.] Are there large gaps between genes in collinear blocks?
\item [$Q_{14}$.] How can $Q_{10}$-$Q_{13}$ be investigated when a collinear block is reversed?
\end{enumerate}

There are also several other analysis tasks that researchers perform that go beyond the block level into the nucleotide level. However, due to the limitations of the underlying data, we ignore these tasks as they are beyond the scope of this project, and there are several other systems specifically designed to investigate collinearity at the nucleotide level such as JBrowse \cite{skinner2009jbrowse}.


\subsection{Supplementary Design Requirements}
Apart from facilitating a system that can let researchers answer $Q_1$-$Q_{14}$ we also explore other design requirements that were raised during our discussions with our research collaborators. 

\textbf{Data Refinement} The underlying data that answers questions $Q_1$-$Q_9$ can be noisy due to the wrong parametric values used during synteny detection and can be filtered further based on several data attributes of the conserved blocks such as the similarity score, gene count and expect value (\textit{e-value}). This requires an interactive feature in the visual design of the system that can assist researchers in extracting a subset of the original dataset based on a particular threshold on a specific data attribute. Another direction in which this requirement can be extended, is having the ability to search for a specific gene in a dataset to look at all collinear blocks containing the gene.

\textbf{Data Enhancement} Analysis Tasks $Q_1$-$Q_9$ can be improved by providing a secondary data series that can contain additional information about the original sequences such as gene density and SNPs. This can also make it easy for researchers to identify patterns relative to known genomic markers.

\textbf{Data Exploration} Researchers need to have a system that offers multiple visual representations of the same information instead of relying on a single visualization to explore the richness of the data, shown by the different data attributes such as size, location , proximity and orientation of the conserved regions.

\textbf{Revisitation Support} The ability to simultaneously refine, enhance and explore complex data visualizations during synteny analysis can often overload the spatial cognition of the user causing them to lose context of their location in their workspace. This needs to be avoided and users need to be given a way to revisit the various steps in their synteny analysis pipeline when exploring a complex dataset.

\textbf{Web Access} In an increasingly connected world where research happens between a diverse set of collaborators spread across the globe, there is a rising need for synteny analysis systems to be available as web applications. Researchers need the final system to be accessed through a website that can let them directly upload the results of synteny detection tools to visualize the required information.





\chapter{Visual Design}

Visualizing syntenic data is a multi faceted problem as not only can the visual representation change based on the underlying biological question but also the resolution at which the data is being visualized.To solve this problem we adopted the taxonomy of design space suggested by previous synteny visualizers like Mizbee \cite{Meyer2009} and added in our own recommendations to arrive at two basic kinds of visual representations for genomic conservation each with its own merits and demerits.We built a linked composite view where users could see both the views simultaneously and interact with them.We also adopted the visual information seeking mantra of overview,zoom and filter for exploration of data across multiple scales by presenting the visualizations in multiple tiers. 

\section{Visual Encoding}

A common way to represent sequence alignment or similarity is to visualize it as a two dimensional `dot plot' \cite{SONNHAMMER1995GC1,Cabanettes2018} through positional encoding.We adopted this strategy for our first visual representation by placing the source and target genomes along the \textit{x} and \textit{y} axes respectively and marking gene alignments with dots as shown in Figure \ref{fig:ch_4_dot_plot_a}.Grid-lines were then further added to the plot to indicate chromosomal boundaries.

\begin{figure}[h]
  \centering
  \includegraphics[width=.475\linewidth]{images/ch_4_dot_plot_a.PNG}
  \captionof{figure}{Dot plot showing whole genome synteny between Oriza sativa(rice) and Sorghum bicolor(corn) with grid-lines added for chromosomal boundaries.}
  \label{fig:ch_4_dot_plot_a}
\end{figure}

This plot can also be adopted for other resolutions by changing the genomes along \textit{x,y} axes to either individual chromosomes or smaller gene blocks.Such matrix based representations are very good at providing an overview of the dataset and can be used to highlight breaks, inversions and duplications as shown in Figure \ref{fig:ch_4_dot_plot_b}.However being a fairly primitive visual representation dot plots are often found to be visually unappealing and complex to understand without the proper background context making them unsuitable for scientific publications compared to other alternatives.


\begin{figure}
  \centering
  \includegraphics[width=.475\linewidth]{images/ch_4_dot_plot_b.PNG}
  \captionof{figure}{Dot plot showing breaks,inversions and duplication events between chromosome 2 and 4 of Oriza Sativa(rice) and Sorghum Bicolor(corn) respectively.}
  \label{fig:ch_4_dot_plot_b}
\end{figure}


For our second basic visual representation we adopt a design that represents synteny through a combination of positional encoding for genomic distances and connected lines for similarity.In this approach genomic sequences are stacked horizontally and similar genes are connected through lines to indicate similarity.However, unlike dot plots which use the same visual encoding across all genomic sizes for this visual representation we adopt a different secondary encoding based on the resolution of the genomic sequences being visualized. 


There are three basic levels in which synteny can be visualized starting from the gene block level which is the smallest unit at which syntenic data is reported.It is a collection of collinear genes in the source genome that are aligned to a group of collinear genes in the target genome.To encode conservation at this level, two gene blocks are represented by line segments that are stacked parallel to each other and similar genes within the blocks are connected with ribbons as shown in figure.The length of the connecting ribbon is based on the number of base pairs in the matching genes.The source and target gene blocks are annotated with numeric tracks corresponding to their position in the chromosome and are coloured in distinct colors to distinguish them.The individual genes are highlighted with a deeper shade of the base colour of the track for easier reference. 

\begin{figure}
  \centering
  \includegraphics[width=.50\linewidth]{images/ch_4_link_plot.PNG}
  \captionof{figure}{Link Plot at the Gene-block Level}
  \label{fig:ch_4_link_plot}
\end{figure}



At the next level individual chromosomes are considered since gene blocks are internally smaller segments within a chromosome.Visualizing synteny at this level involves encoding information related to the location,size and orientation of conserved regions.To achieve this chromosomes are stacked parallel to each other and their lengths are encoded to reflect their genomic size and so chromosomes with more base-pairs in them show up as longer line segments.conserved regions in the chromosomes are then connected through ribbons from their positions on the chromosome to indicate similarity.This encodes both the location and the size of the of the conserved regions as the width of ribbons changes based on the genomic size of the linked gene blocks.


\begin{figure}[h]
  \centering
  \includegraphics[width=.50\linewidth]{images/ch_4_link_plot_chromosome_a.PNG}
  \captionof{figure}{Link Plot at the Chromosome level where the blue coloured ribbons represent forward matches and the red coloured ribbons represent reverse matches(inversions).}
  \label{fig:ch_4_link_plot_chromosome_a}
\end{figure}


To encode orientation of the gene block secondary encoding in the form of colour is adopted to visual distinguish gene inversions as shown in figure \ref{fig:ch_4_link_plot_chromosome_a}.So forward matches are coloured in blue and reverse matches are coloured in red.Unlike the gene block level, at the chromosome level several bands can overlap and cross each other due to multiple gene translocation and inversions events and can cause visual clutter.To mitigate this problem complex polygons are used instead of rectangular ribbons and are generated through \textbf{B}-spline curves\cite{ref851370272} with control points set to bundle the curves towards the centre.The control points points are set vertically in the middle of the parallel blocks to ensure that the original size of the ribbons remain undistorted at regions where they join the chromosome as they visually represent the size of the conserved region as shown in figure \ref{fig:ch_4_link_plot_chromosome_b}. 

\begin{figure}
  \centering
  \includegraphics[width=.50\linewidth]{images/ch_4_link_plot_chromosome_b.PNG}
  \captionof{figure}{Ribbon bundling to reduce visual clutter with the control points set towards the centre indicated in a single gene-block.}
  \label{fig:ch_4_link_plot_chromosome_b}
\end{figure}


Finally at the whole genome level where syteny is observed between several chromosomes at once, smaller details such as inversions have less precedence and so the secondary encoding in the form of colour is used to distinguish different chromosomes instead of the orientation of the syntenic region.A layout similar to the parallel stacking at chromosome level is adopted however instead of having a single connected unit for the entire genome chromosomes are separated from each other with gaps serving to indicate the start and end of each chromosome.Chromosomes in the source layer are assigned a unique color while chromosomes in the target layer are assigned an alternate gray coloring scheme.Ribbons are then linked between conserved regions to represent syntenic gene-blocks and are assigned a color based on their source chromosome.This form of encoding location information about the source in the connection through colour has been used earlier in other syteny visualisations systems and has been proved effective \cite{Meyer2009}.We adopt the aforementioned bundling strategy of using \textbf{B}-spline curves\cite{ref851370272} to improve visual clarity but set the control points independently for every chromosome to group all the gene blocks emerging from each chromosome into a single bundle.


\begin{figure}
  \centering
  \includegraphics[width=.75\linewidth]{images/ch_4_genome_level.PNG}
  \captionof{figure}{Visual encoding at the chromosome level with connecting ribbons coloured based on the source chromosome they are linked from.}
  \label{fig:ch_4_genome_level}
\end{figure}



\section{Layout Strategies}

A common strategy that is used among all the three parallel stacked representations is the vertical separation between the source and the target to visually distinguish the two regions.This is easy to implement at the gene-block level and the chromosome level as the source and target regions are single continuous entities but requires minor adaptations at the genome level.The genome is a combination of several chromosomes and so each chromosome had to be individually distinguishable while still being represented as a part of the whole source group and different from the target group.To achieve this grouping we use the visual law of proximity from Gestalt principles \cite{wertheimer1923untersuchungen} and represent each chromosome as a pill shaped region and then lay them out end to end horizontally with small gaps between them.The gaps between the chromosomes achieve the task of making the chromosomes look distinct and also being smaller than vertical gaps between the two genomes, cluster the source and the target regions into two distinct groups visually.

\begin{figure}
  \centering
  \includegraphics[width=.75\linewidth]{images/ch_4_layout.PNG}
  \captionof{figure}{Different layout strategies at the genome level with conservation being encoded as connections.}
  \label{fig:ch_4_layout}
\end{figure}


In arriving at the optimal layout strategy we looked at several different alternatives ways of arranging the chromosomes.In the popular syteny browser MizBee\cite{Meyer2009} the authors provide a taxonomy of the different synteny layouts and broadly classify them into two categories: contiguous and discrete.In the former the chromosomes are presented adjacent to each other either in a linear or a circular layout and in the later chromosomes are treated as distinct elements and presented either in segregated groups or interleaved with each other.In our design we go for the contiguous scheme but we omit the circular layout as it has already been explored in AccuSyn\cite{accusyn}  and instead look at possible linear layout strategies where conservation is encoded through connections as shown in figure \ref{fig:ch_4_layout}.In the vertical(a) and horizontal layouts(b) the underlying strategy is similar except for the orientation of the two parallel layers.However the number of chromosomes in a genome can be numerous as in the case of humans who have 23 and thus cause the vertical layout to be quite long which makes it sub-optimal for most common screen aspect ratios.Thus of the two, the horizontal parallel layout is the preferred mode of encoding synteny.A common advantage of these two layouts is that they can stacked in multiple layers such that chromosomes at every level are linked to both chromosomes above them and also the chromosomes below then.This stacked layout strategy is used to represent syteny in the form of a tree view chart and can be particularly useful in scenarios where conserved regions need to be traced across several evolutionary levels.

The bi-directional linking strategy is however unavailable in the parallel layout scheme in pairwise comparisons scenarios but can utilized by moving the two layers adjacent to each other as shown by the layout(c) in Figure \ref{fig:ch_4_layout}.In this layout the chromosomes are in the same level thus making it possible for conserved regions to be linked in two directions either from the top or from down below.This gives us an opportunity to add an additional layer of encoding.For example if we had to represent the orientation of the conserved regions we could link all forward matches through connections from top and all reverse matches through connections from the bottom.The disadvantage of this layout is the high number of crossing between the connections.This can be made worse in scenarios where there is a high degree of collinearity between the two genomes due to the ordering of the chromosomes as every connection between the first chromosome in the source and the first chromosome in the target is crossed by all other connections emerging from the rest of the chromosomes in the source.An alternative approach to solve this problem includes reversing the layout of one of the layers or arranging the chromosomes in a radially outwards fashion in  both the layers.This layout can also be extended to express synteny in multiple levels by simply increasing the number of radial layers such that each layer is connected to both the layer on its right and the layer on its left.

\begin{figure}
  \centering
  \includegraphics[width=.95\linewidth]{images/ch_4_layout_multi.PNG}
  \captionof{figure}{Multi-level layout strategies.}
  \label{fig:ch_4_layout}
\end{figure}

\section{Design Iterations}

dot plot old and new 

link plot original , then alternate vertical view , 
then seperate into pills use colors ,

then curve them towards

\section{Interaction Design}

why a linked view

another composite view
% refer to paper on linked views and add more lierature 
% State of the Art:
% Coordinated & Multiple Views in Exploratory Visualization
% Jonathan C. Roberts
% Computing Laboratory, University of Kent, UK
% j.c.roberts@kent.ac.uk


% To visualize genomic conservation we adopted a design based on the popular Visual Information Seeking Mantra \cite{Shneiderman96theeyes} which consists of overview first,zoom and filter, then details on demand.Our design presents the information in a top down approach in three distinct levels starting from the whole genome level to an individual chromosome level and ending on the gene block level.At everfry level users are given the option to filter view details on demand by mouse interactions with the visual elemenets

(end this chapter with a discussion of the visual choices for the filter panel)

\chapter{SynVisio}

Our biggest contribution in this research was the development of SynVisio an online platform to explore syteny by mapping syntenic blocks that are highly conserved and long enough to be significant between a given pair of genomes or within a single genome.In this chapter we first discuss the different modes SynVisio offers for synteny analysis and how each one operates.We then explore the different features SynVisio provides to enhance user experience with the tool.Following this we discuss the software implementation of the tool and the elaborate on choices made regarding its web architecture.Finally we look at a basic usage scenario through a series of steps depicting how SynVisio can be used to explore the genomic conservation between a sample data-set of humans and chimpanzees.

\section{System Overview}
SynVisio is a multi scale genome browser that can be accessed through the web and lets researchers explore genomic conservation.It lets researchers upload output files of synteny detection system of their choice and generates visualizations on demand from the information in these files.It offers two analysis modes : Single Level and Multi Level.In the first mode users can compare genomes two at a time through a dashboard where syteny is visualized as both a dot plot and a linked parallel plot.The charts are accompanied by a filter panel where the conserved genomic blocks can be filtered based on features such as the degree of similarity.In the second mode researchers can compare several genomes at a time through multi level representations such as hive plots and stacked parallel plots.To aid researchers in their visual exploration of synteny,SynVisio lets them annotate the generated charts with additional tracks in the form of histograms, heat-maps and other basic plots.Additional features are also provided such as a gene search panel to look for specific genes by gene ID and the ability to export generated charts for publication.


\section{Analysis Mode}
Gene sequences can be compared in different ways depending on the underlying biological question.Which means syteny analysis can vary between visualizing simple pairwise matches between two genomes to preforming multi way comparisons across several genomes at once.The availability of datasets and their inherent quality also plays into the kind of analysis that can be done.Whole genome alignment for example is usually done pairwise as looking for matches can be faster when the subset of available matches is low.Additionally in the context of Synteny detection which is anchor   based, identifying common markers between multiple genomes is difficult\cite{wang2012mcscanx}.However when the data is available, multi way comparisons can offer better insights and tackle bigger questions like pan genome syteny.Thus SynVisio offers both a Single and a Multi Level analysis mode depending on the the researchers choice and the availability of data.

\subsection{Single Level Analysis}
This is default mode in which SynVisio operates and is meant for exploratory tasks as it presents the collinearity between a selected set of chromosomes in two different visual representations and lets users filter the collinear blocks on the fly.Although our system operates as dashboard with multiple representations in coordinated views we offer users the ability to  look only at one particular representation through the configuration page.This is meant to make our system unopinionated in the choice of representation and lets users make the decision on how they want their data visualized. 

\begin{figure}
  \centering
  \includegraphics[width=1\linewidth]{images/ch_5_baseparameters.PNG}
  \captionof{figure}{Synteny detection parameters and level of collinearity presented along with toggles to select source and target chromosomes.}
  \label{fig:ch_5_baseparameters}
\end{figure}


The first step involved in using the dashboard is providing an input dataset, for this, users can either upload their own datasets or use existing sample files.We have already processed several datasets depicting genome conservation on a wide range of species.These are available on the homepage of our application and are updated on a monthly basis.Some of the examples include self syteny in Brassica napus(canola), cross syteny between Oriza satica(rice) and Sorghum bicolor(broom-corn) and cross syteny between Arabidopsis thaliana(thale cress) and Vitis vinifera(grape vine).After the initial data uploading and processing stage is complete basic information about the parameters used in the syteny detection process are presented along with the percentage share of collinearity present in the files accompanied with toggles to select the source and target chromosomes as shown in Figure \ref{fig:ch_5_baseparameters}.If outputs of synteny detection systems other than MCScanX are uploaded the parameters tab and the percentage share of collinear genes chart are left blank . The list of chromosomes are ordered alpha-numerically to divide the different species into distinct groups and make it easier to pick chromosomes sequentially.Additional buttons are also provided to either``Select All" or ``Clear All" in both the drop-drown lists intended for picking chromosomes.


When a user hits the button ``GO" to get started the first visualization presented at the top is a parallel link plot where syntenic collinear blocks are connected by coloured ribbons as shown in Figure \ref{fig:ch_4_dashboard}.The source chromosomes are laid out on the top and the target chromosomes are spread out in the bottom.The size of the chromosomes are calculated based on the genomic sizes of the chromosomes and the available screen width to ensure that the visualizations are responsive across different screen-sizes.Chromosomes in the source layer are coloured using a chromatic 10 point color scale derived from ColorBrewer\cite{colorbrewer} and are set to repeat after every 10 chromosomes as humans can find it hard to perceive differences beyond a dozen colours\cite{ware2012information}.The connecting ribbons represent collinear blocks with the colour of a ribbon representing its source chromosome.The connecting ribbons can have varying widths at either ends due to the size of the collinear block they represent. Although collinear blocks have the same gene count at either side the width of the block in terms of base pairs can be quite different at either sides due to variable gap sizes between individual genes.This scaled representation of connected ribbons can also mean that certain ribbons can end up being smaller than a single pixel in width due to their small genomic size.So we clamp our scale at the lower end to 2 pixels to ensure that extremely small ribbons are still represented as 2 pixel wide lines instead of ribbons.

While the parallel link chart is designed to take half of the available vertical space, the other half is made up of a dot plot and an adaptive filter panel consisting of a scatter plot.The dot plot as explained in the visual design chapter uses positional encoding and represents collinear blocks as either dots or lines.To ensure that extremely small collinear blocks are still represented on the screen we limit them to single pixel wide dots on the chart while larger blocks are encoded as lines.The dot plot works in coordinated manner with the parallel link plot and so any actions in one are also reflected in the other as shown in figure {cite linking view}. Since the dot plot is always meant to be square it has a fixed aspect ratio and thus the filter panel expands to fill the remaining horizontal space.It provides filtering through three parameters : Gene count, Match Score and E(expect) value.It is set to filter using gene count by default but can be changed using the radio buttons provided to the left.To offers users context into the parameter being filtered its value for all the collinear gene blocks are visualized as a simple scatter plot.Every collinear gene block is represented by a single dot irrespective of its size  and is colour coded to correspond to the chromosomes in the parallel plot.The scale of the plot is adaptive and automatically changes based on the parameter in question.Gene count and Match Score correspond to the number of genes in a collinear block and the alignment score assigned to that block respectively and are represented in a linear scale.E-value or expect value is the measure of probability that a match has occurred by chance and owing to the wide range in which this value can be reported it is represented in a logarithmic scale.Researchers can use the slider to control the visibility of collinear blocks they see in the other two views.The position of the slider on the chart is represented with a dashed line that is annotated with the value at which the charts are currently being filtered.

\begin{figure}
  \centering
  \includegraphics[width=.75\linewidth]{images/ch_1_dashboard.PNG}
  \captionof{figure}{Single analysis mode visualizing genome collinearity in Bn(Brassica Napus) with the following components: \textbf{a)} Parallel Link Plot \textbf{b)}Dot plot and \textbf{c)} Filter panel}
  \label{fig:ch_4_dashboard}
\end{figure}


% For uploading their own datasets users need to provide two files, a collinearity file that is generated using a syteny detection software like MCScanX or DAGChainer containing the list of collinear blocks and a GFF file for the positions of all the reference genes in the genome.


% In this Mode SynVisio operates as a dashboard by default and presents genome collinearity in multiple representations.Users however have the choice to switch the visual representation using the configuration page and options available are a dot plot,a linear connector plot, and an exploratory dashboard with both plots and a filter panel.


% First talk about parameters and overall percentage share between source and target. and why its shown
% then filter panel why we select chromesome differe source and target. why they are sorted apla nuerically

% then we talk about the linear link plot then dot plot.
% then the filter panel why we need it..why the different features. why the slider. choice of colours at this stage.

% then chromosome mode .different colour for inversions then the ability to zoom and hover over links to see what they contain.

% then individual block mode then invert entire chromosome. then shift axis to right or zoom.



\subsection{Multi Level Analysis}
Hive Plots and Stacked Bar plots

\section{Usability Features}
\subsection{Track Annotation}
\subsection{Re-visitation Support}
\subsection{Gene Search Panel}
\subsection{Image Export}


\section{Usage Scenario}
A series of pictures explaining a particular exploratory scenario - Best done with Wheat genome showing genome triplication and further interlocation of some chromosomes.
\chapter{Evaluation}

SynVisio was made publicly available to use for free on the Internet on September 2018. A stable version was deployed in the start of 2019 and since then, it has been used by several researchers across the world in exploring genomic conservation in a wide variety of organisms. Evaluation of the system was done through a combination of user study based on semi-structured interviews, and analysis of web traffic to the system. Domain experts from biology were consulted for the user study and their feedback of the system is summarized through three case studies presented in the sections below. To provide evidence on the effectiveness of our system in the wild we also explored user activity logs on the website for a period of 12 months through google analytics. Finally to demonstrate the open-ended design of SynVisio we provide examples of genome databases for silkworm and two other species that were extended to show synteny visualizations using the open-sourced code of SynVisio.

\section{Methods}
The user study to evaluate our system was conducted in the form of semi-structured interviews with 5 domain experts from three major research groups we were collaborating with; one of the experts was a bioinformatician who worked across all three research groups. The interviews were conducted either through phone or in person and lasted around 45-60 minutes. All domain experts had considerable exposure to SynVisio and were familiar with the different features provided by the system. This was further verified before the interviews were conducted. Researchers were first asked about the relevance of synteny visualizations in their field of research and then asked to give their opinion on the different modes of analysis that SynVisio offered through 3 open ended questions with one question targeting each genomic resolution (genome, chromosome and gene level). They were then asked to give feedback for the different interactive features provided in the system. Finally all researchers were asked to rate SynVisio on a scale of 1(very bad) to 5(very good) for its ability to represent genomic conservation.


\section{Case Studies}

\subsection{Wheat\textit{(Triticum aestivum)}}
Wheat is one of the most widely cultivated crops in the world and plays an important role in human nutrition. Being a common cereal, wheat genomes are highly diverse and spread over a large geographic range. The genome is capable of tolerating mutations and extensive hybridization which is why it has been able to adapt to such a wide variety of environmental conditions\cite{wheatinfo,10wheat}. Wheat is also one of the \textit{neolithic founder crops} that were the first to be domesticated almost 10000 years ago.
Bread wheat(AABBDD) is a hexaploid genome and is the result of series of hybridization events between three ancestral genomes A Donor\textit{(Triticum urartu)}, B Donor\textit{(Aegilops speltoides)} and D Donor\textit{(Aegilops tauschii)} which makes it an interesting subject for synteny analysis. Our collaborators were part of a research team involved in sequencing a high quality version of the wheat genome. Since the wheat genome is extremely large and complex, synteny analysis can help researchers in assessing the quality and contiguity of the genome assembly though alignments between the sub genomes (A,B and D).

Our collaborators relied on visualizations generated by SynVisio to present and summarize their sequence assembly results - \textit{``the images have been used in presentations, academic meetings such as the international wheat congress and also at the plant-animal genome conference.'' (R1)}. While they used Circos style plots for research publications earlier, they mentioned that the multi level representations in SynVisio were far more useful for genomes like wheat with many chromosomes - \textit{``This tool is better than a Circos plot, especially when comparing multiple genomes, circos can be limited because you are seeing too many chromosomes in one circle and so are losing information ... a stacked layout like yours is easier to see.'' (R4)}. One collaborator in particular also appreciated the system for its ability to handle large datasets like wheat - \textit{``this is really neat. this is also very useful...a single wheat chromosome is vast and wheat has 21 of those (chromosomes) placing stress on an analysis pipeline in terms of computational complexity...it is also very repetitive...'' (R1)}.
Feedback provided by this research team was also used in adding support for hive plots which offer an intuitive representation to compare multiple sub genomes in crops like wheat (see Figure \ref{fig:ch_6_wheat}). The Appendix provides supporting material describing the process needed to generate this hive plot using SynVisio for this particular dataset. Our collaborators in this team plan on publishing images generated using SynVisio and have already used it to create a portal for researchers to compare 12 different wheat cultivars for the 10+ Wheat Genomes Project. Users can use this portal to select any two varieties from 12 different wheat cultivars and then compare genomic conservation between them using SynVisio\cite{10wheat,wheatinfogithub}.

\begin{figure}
  \centering
  \includegraphics[width=0.65\linewidth]{images/ch_6_wheat.PNG}
  \captionof{figure}{Genomic conservation between the three sub genomes A,B and D of Wheat (Chinese Spring Variety) shown through a Hive plot in SynVisio.}
  \label{fig:ch_6_wheat}
\end{figure}


\begin{figure}
  \centering
  \includegraphics[width=1\linewidth]{images/ch_6_lentils.png}
  \captionof{figure}{Collinearity between Lentils (Lc), Barrel Medic (Mt) and Chickpea (Ca) presented through a Tree view plot. The ordering (Ca) and orientation (Mt8, Ca4, and Ca6 - flipped) of some chromosomes have been changed to reduce visual clutter.}
  \label{fig:ch_6_lentils}
\end{figure}


\subsection{Lentils\textit{(Lens culinaris)}}
Lentil is an important legume crop that is grown globally as a valuable source for dietary protein. It also plays a crucial role in food security in developing countries along with other legume crops like Chickpea \textit{(Cicer arietinum)}\cite{varshney2013draft}. Lentils can be made more resistant to diseases and weed infestations by increasing the genetic diversity of the genome through hybridization between disease resistant wild varieties. This however requires mapping the traits through molecular markers to assess their diversity. Our collaborators relied on comparative genomic mapping to leverage information from a model legume species like Barrel Medic (Medicago truncatula) onto less studied crop species like lentils and chickpea due to their lack of common markers.

Unlike the wheat genome, synteny analysis requirements for this project were centered around cross synteny between species rather than self synteny. Due to the large size difference in the genomes between Lentils (4Gbp) and Chickpea (740Mbp) the first version of SynVisio was not able to generate legible charts as the Lentil chromosomes were extremely wide compared to chromosomes from the other species and so a special feature was added to have variable scales at different levels. Our collaborators were pleased with the updated view and also remarked on the multiple visual representations provided in the genome view - \textit{``I think it's quite good, I do really like that there's also the dot plot, in the corner, so that if anything is a little bit unclear, from the parallel view, you can kind of refer back to that.
'' (R5)}. Because this was a cross synteny analysis between several genomes, researchers mentioned that the Tree view was particularly helpful in summarising large scale chromosomal rearrangements and inversions while still keeping the different genomes visually distinct as shown in figure \ref{fig:ch_6_lentils}. They also compared it to circos plots and remarked on its usability - \textit{``It's like the circos plots are beautiful but you can't do anything with it. Whereas this, the tree-view in particular, is very aesthetically pleasing and that's the kind of thing that you can show to your collaborators and you can also understand it, at the same time, and then the interactive nature of it helps too...
(R5)''}. Reseachers from this group have also used SynVisio to study genomic conservation in other legumes like the Tepary Bean \textit{(Phaseolus acutifolius)} and are planing on using it to generate images for their research publications in future.

\subsection{Canola\textit{(Brassica napus)}}
Canola is an important oil seed crop in the world as it is an excellent source for both animal feed and high quality edible oil\cite{shahidi1990canola}. It is an allotetraploid (4 copies - AACC) species that was formed through interspecific hybridization between diploid ancestors Brassica rapa (A Donor) and Brassica oleracea (C Donor)\cite{parkin1995identification}. Studying this genomic conservation can help researchers in looking at genetic variations that are advantageous from an evolutionary perspective in polyploids like Canola. Our collaborators from this research group were particularly interested in using comparative mapping to understand the level of genome duplication in modern brassica cultivars and the occurrence of genomic rearrangement in the evolution of these varieties from a common ancestor. This meant that they needed to visualize both self synteny between Canola itself and also cross synteny between canola and its closely related species. - \textit{``...in polyploid plants where there are many genomic rearrangements, visualization is really useful because there is lots of information and its really complicated for us to understand without an overview... (R2)''.}

SynVisio also helped researchers in this team at refining their assemblies - \textit{``Our assembly got better when we upgraded our sequencing from short read to long read sequencing technology as more regions are assembled. This tools helps us visualize that improvement ...(R2)''}. Regarding the visual representations one collaborator remarked that the parallel plot representation was better at showing genomic conservation than dot plots - \textit{``We have always used dot plots but these (parallel plots) are visually more intuitive...when chromosomes start breaking apart its much more difficult to follow where things are going in that big square and in this its easier to play around...Its much easier to trace things and work out where you are...(R3)''}. Researchers from this team have used visualizations generated by SynVisio at several conferences such as PAG (Plant Animal Genome) 2019 \cite{brassicapag} and also in a recent publication describing long read assemblies of two diploid Brassica species\cite{perumal2020high}.
 
\section{Global Usage Analysis}

Although our system was initially designed based on requirements from our collaborators for use within our university it was made publicly available as an open access tool on the internet and has been used since then by researchers across the world in a wide variety of research projects and images generated by SynVisio have also been used in research publications describing new genome assemblies and annotations\cite{mathers2020improved,perumal2020high}.

To quantify the use of SynVisio since it was made public we analysed web traffic through Google analytics  from Jan 1st 2019 to Jan 1st 2020. In this period of 12 months SynVisio had 154 unique users spanning 267 sessions with an average session duration for each session being around 2 minutes while some users spent as much as 28 minutes on the system exploring different datasets. Users were from 18 different countries across the world as shown in Figure \ref{fig:ch_6_users} with a majority of the users being from China(53) followed by the United States(45) and Canada(23).

\begin{figure}
  \centering
  \includegraphics[width=1\linewidth]{images/ch_6_users.PNG}
  \captionof{figure}{Global user distribution of SynVisio for a period of 12 months from 2019-2020.}
  \label{fig:ch_6_users}
\end{figure}


SynVisio was designed in a modular fashion as a reusable component and its source code was open-sourced through an MIT license on GitHub\cite{synvisio}, which meant that it could be adopted and reused in other research tools and projects. An example of a system that has used SynVisio in this manner is \textbf{TeaBase}, an online genome database with various tools, one of which is SynVisio, to explore the Tea plant(\textit{Camellia sinensis}) genome as shown in Figure \ref{fig:ch_6_other}\cite{teabase}. Other genome databases that have used SynVisio in a similar manner are \textbf{VitisGDB} and  \textbf{SilkDB 3.0} to explore the genomes of Grapevine and Silkworm respectively\cite{lu2020silkdb,vitisgdb}.

\begin{figure}[h]
  \centering
  \includegraphics[width=1\linewidth]{images/ch_6_other.PNG}
  \captionof{figure}{TeaBase, an online genome database to explore the Tea plant genome with a modified version of SynVisio adapted into its toolset. }
  \label{fig:ch_6_other}
\end{figure}


\section{Evaluation Summary}  
Although several visualization tools exist for analysing genomic conservation they are not easily accessible as mentioned by researchers we interviewed -  \textit{``There are a couple of R based tools that we use but none of them are as complete as synvisio (R4)''}. SynVisio has been able to fill this critical gap
- \textit{``There isnt anything like this. Especially not where you can play around with your dataset. (R3)''}. When asked to rate SynVisio for its ability to visualize genomic conservation across different levels on a scale of 1(very bad) to 5(very good), four researchers gave the system the highest rating of 5/5 and one researcher gave it a rating of 4/5 stating that they would have liked greater control over the ordering and orientation of the chromosomes. This feedback validates the usability of the system for the initial set of visual tasks we designed the system for. Further we were also able to meet the supplementary design requirements that we had envisioned for data refinement and enhancement. Although some researchers mentioned that they did not find the filter panel quite helpful \textit{``I do most of my filtering ahead of time before running the tool so this filter is personally not useful for me but I can see why you have it...
 (R4)''} others found use for it - \textit{``The images are quite messy and it (filter panel) is definitely helpful in cleaning it up a bit... (R3)''}, \textit{``When looking for distant relatives, the feature with E value filter is useful.(R2)''}. 
 
Making SynVisio an online tool with the ability to directly upload sequences to visualize them has improved the usability of the tool to a great extent as several researchers mentioned that this has allowed them to share their work with other collaborators easily -\textit{`` if I wanted to discuss some of the results of this with a collaborator, I just zip up the two files that you need for input. Send it off to them and they can put it in and play with it themselves(R5)''}. This is also further evident by the web traffic SynVisio has received since its has been made available on the internet. Further the system design has also been adopted in several online genome databases, showing that it is a valuable tool in exploring genomes and interacting with them.




\chapter{Discussion}

In this chapter we discuss the the insights gained from building each of the unique features of our system and the design choices that went into their development. We also look at how some of these features can be extended to support additional genomic analysis tasks. We then explore some of the limitations of our current system based on the feedback gathered from our user evaluation study and possible improvements that can be made in the future.

 \section{Design Implications}
 
\begin{itemize}
    \item \textbf{Input Files and Formats} - Genomic conservation can be detected through a wide range of tools which means that it can be represented in a wide variety of file formats depending on the type and level of information about conservation of gene order.
    Although some tools like Mizbee have relied on users to supply input in a standarized format, this is not a viable solution as this often means users will have to rely of a custom script to transform their analysis files into the required format. Visualization tools like SynVisio and Mizbee are part of larger ecosystem of genomic analysis tools and so they need to offer at least partial connectivity between such tools which means outputs of most analysis tools should be directly supported in visualizations tools without the need for intermediate processing. In an effort to address this developers should consider building systems that offer support for heterogeneous data. For example, SynVisio currently supports inputs from several popular tools such as collinearity files generated by MCScanX or Orthologous files generated by Dagchainer with also partial support for MUMmer output files. 
 
    \item \textbf{Web Accessibility} - Most existing genomic visualization tools are desktop applications or packages in languages such as R,Python or Perl, however, there has been a gradual shift towards the web in the recent years. Although desktop applications are efficient at utilizing system resources they are limited in their accessibility as they are not often supported in all operating systems and require extensive customization from developers as they are platform dependant. Web applications on the other hand are platform independent and can be built once and used everywhere thus requiring less development effort. Even though web apps are limited in their processing capability they can rely on remote servers for intensive processing and some applications like SynVisio also rely on web workers to process data in parallel threads for more efficient data processing. This easy accessibility and low cost maintenance of web apps coupled with support for collaborative work means that web applications should be the first choice for developers of visualization tools in future.
    
    \item \textbf{Multi Layout and Multi Scale Views} - 
    Genomic data can be analysed at multiple resolutions and the visual representations vary at each level. At the genome level visual representations are chosen to emphasize  approximate position of the conserved regions and their chromosomal identifiers. This can be useful in scenarios such as, during genome assemblies when errors or breaks can be easily identified. However when looking at chromosome level orientation of the conserved region is also highlighted and finally at the individual gene level emphasis is placed exclusively on the order of collinear genes and their exact function and location in the genome. Visual representations can also vary based on the task at hand, for example, dot plots are a popular choice for summarizing large scale datasets as they offer a compact representation of both the position and orientation of conserved regions. However, their orthogonal representation is difficult to understand making them an unpopular choice for browsing and locating selective conserved regions in comparison to parallel plots. Other such examples are stacked parallel plots which are good at tracing collinear regions across several genomes. Thus in designed genomic visualization systems developers should rely on a combination of visual representations or provide users the choice to switch between different representations based on the task at hand. 

    \item \textbf{Visual Navigation and Linked Views} - When exploring genomic data visualization systems can provide users several ways to traverse the different representations at each genomic scale. However the most intuitive way to explore the dataset would be to start at the genome level and drill down all the way into the individual gene level and so visualizations at each level should be provided with interaction techniques to filter and zoom into a particular part of the dataset which can then be viewed in a different visualization at the next inner level. This form of a tiered navigation combined with the support for revisitation at any point can help researchers in easily going back and forth between the levels and exploring a large possible number of scenarios without losing context. Also a major part of analyzing genomic conservation involves comparison and providing linked multi views that are different representations of the same information can help users in contextualizing the conservation and better understanding it.
    
    \item \textbf{Linear vs Non Linear Representations} - Genomic data is usually linear in nature and so the best form of representing such data would be a linear representation through a dot plot or a parallel plot but circos style plots which user non linear representations are still quite popular in genomic visualizations as they are aesthetically pleasing and offer a compact picture. However based on observations from our user evaluation, several users found circos plots difficult to navigate for an indepth analysis. This is primarily due to the non linear representation of these plots than can make it difficult to identify connections between distinct groups which in this case are chromosomes placed in a circular layout. Also linear representations like parallel plots can be stacked on top of each other to represent conservation between multiple genomes but circular layout can only handle a limited number of chromosomes in the central layout before they become difficult to comprehend due to close proximity between each other. Part of the compact nature of the circos style plot arises from the ability to stack several circular layers on top of each other to represent several tracks, however the radial nature of this design means that tracks in the outer layers are always larger than the tracks in the inner layer. This can lead to visual bias where patterns in the outer layers are more prominent than patterns in the inner layers.
    
    \item \textbf{Adaptable Genome Scales} - Genomic data can be incredibly diverse in size and so systems visualizing such information need to automatically adapt to different scales of data instead of relying on a standard scale. Some plant genomes like wheat are extremely large (17 Gigabases) and quite dispersed due to which the genes are quite small and so when visualizing this information, SynVisio provides users 2 additional levels to magnify the dataset. Similarly when visualizing extremely small genomes such as viral genomes (30 Kilobases) the system automatically loads up the data in the smallest possible resolution directly at the gene level as shown in Figure \ref{fig:ch_7_viral} which compares similarity between two coronavirus strains that led to global pandemics. This disparity in genome sizes can also be an issue when comparing multiple genomes with a wide difference in their sizes, in such scenarios the visualization should provide users an option to have different scales for each of the genomes instead of relying on a single normalized scale among the genomes. 
\end{itemize}

\begin{figure}[h]
  \centering
  \includegraphics[width=1\linewidth]{images/ch_7_viral.PNG}
  \captionof{figure}{Extensive similarity between genomes of the SARS Virus (2003 pandemic) and the COVID-19 Virus (2019-2020 pandemic) with the RNA responsible for a single protein highlighted in a darker shade.}
  \label{fig:ch_7_viral}
\end{figure}


 \section{Limitations and Future Work}
 
 Although SynVisio was designed to handle a wide variety of scenarios there are still certain issues with our system that can be improved through additional changes in the future. 
 
 \begin{itemize}
    \item The first and major limitation of our system lies in the dependence on an external tool (MCScanX, DAGChainer or Mummer) to detect conserved regions. This was mentioned as a bottleneck by several of our users in analysing their datasets. But the complexity involved in detecting similarity between two given sequences and running a collinearity detection software cannot be achieved through the existing web system as it is computation intensive. A possible way in which this can be solved in the future is by setting up a dedicated remote server that can accept sequences uploaded by users to perform synteny detection on the cloud and then send the results back to the web system to be visualized. This would also gives users a greater flexibility in the changing the different parameters involved in detecting conserved regions such as the E value  to look at more distant matches.    
    
    \item SynVisio is currently opinionated in determining the visual scale of the genome. Genome scales are  calculated based on the available screen width and the size of the genome in base pairs and every chromosome is normalized accordingly. But users are given the option to override the normalization and have independent scales for each genome when comparing multiple genomes. This can lead to confusion when two genomes are stacked parallel to each other and both stretch to fill the available width and users lose context of the size of the genomes. To address this confusion in future we can provide information in the form of tracks or scales indicating the size of the genome in Kilo-bases or Mega-bases.
    
    \item SynVisio automatically sorts chromosomes alphanumerically in each genome to determine their layout. Chromosomes in each genome are presented from left to right and oriented in the same direction. However in some cases this layout can cause the ribbons connecting conserved regions to excessively cross each other making it difficult to understand the relation between the two genomes. In such scenarios it would be helpful if users are provided an option to declutter the layout by reordering the chromosomes or reversing the orientation of each chromosome. This can achieved in the future by developing a dedicated layout editor that lets users manual drag chromosomes around and reverse them if needed to create a more organized layout.
    
    \item Pan genome synteny
    \item Linking into genomeDB, NCBI to provide tighter intetgrations or link out to NCBI.
        


\end{itemize}
 

\chapter{Conclusion}

Comparative genomic research plays a vital role in studying genome evolution and ancestral genome reconstruction. However, despite the availability of high-resolution genomic data, research in this field is being restricted due to the lack of proper analysis tools. While some analysis tasks can be automated to deal with the high volume of data, other tasks still require manual interpretation such as synteny analysis. Visualizing data in such scenarios can help researchers in their analysis by offloading part of the cognitive load required in processing information onto humans' inherent capacity for visual perception. Visualizing synteny blocks can aid researchers in understanding the location, size, and orientation of conserved genomic regions. Although some tools do exist for synteny analysis, they are limited in their usability and offer very little interactivity needed to explore complex datasets. Our primary contribution in this research work is SynVisio, a synteny analysis tool that offers genomic researchers different ways to visualize and explore genomic data. Researchers can access the tool through a public web-based interface and directly upload their synteny analysis files. Information can be analyzed in a primary analysis mode through pairwise comparative visualizations such as linear parallel plots and dot plots. Alternatively, researchers also have access to a multi genome analysis mode where syntenic blocks can be visualized through hive plots or stacked parallel plots to trace genomic conservation across several genomes at once.

Our second contribution was in adding interactive support to our system to help researchers in refining and enhancing their datasets. All visualizations are accompanied by a filter panel to modify the generated visualizations in real time. Syntenic blocks can be filtered based on the level of similarity (score or number of genes in a block) or the probability of the match (E value) depending on the underlying genomic question. Researchers can also augment certain visualizations such as parallel plots and dot plots with tracks representing additional information such as gene density or SNP count. These tracks are in the form of heat-maps, line charts, scatter plots or histograms. The tracks along with all visualizations can also be exported in publication-ready formats.

Our third contribution is providing support for revisitation. Synteny analysis is an exploratory task that requires researchers to investigate conservation at multiple genomic resolutions. Such exploratory analysis requires users to switch between multiple visualizations under different filter parameters. This switching can, however, cause them to lose context of their position in the dataset. SynVisio avoids this by providing users the option to snapshot the state of the system at any point in their exploratory analysis for easy and quick revisitation. This along with other features such as searching for genes in syntenic blocks, can be useful to researchers, particularly in exploring large datasets.

SynVisio has been developed as a modular component that can be reused in existing online genomic analysis tools, and the source code for the system has been open-sourced to facilitate the rapid dissemination of our work into other scenarios. Several researchers are currently using our system across the world either directly via the web interface or through the integration of our system into their existing tools. We also plan on adding additional features to the system in the coming year to offer support for other kinds of genomic analysis tasks.


%%%%%%%%%%%%%%%%%%%%%%%%%%%%%%%%%%%%%%%%%%%%%%%%%%%%%%%%%%%%%%%
% SUBSEQUENT CHAPTERS (or \input's)  GO HERE
%%%%%%%%%%%%%%%%%%%%%%%%%%%%%%%%%%%%%%%%%%%%%%%%%%%%%%%%%%%%%%%

%%%%%%%%%%%%%%%%%%%%%%%%%%%%%%%%%%%%%%%%%%%%%%%%%%%%%%%%%%%%%%%%
% The Bibliograpy should go here. BEFORE appendices!
%%%%%%%%%%%%%%%%%%%%%%%%%%%%%%%%%%%%%%%%%%%%%%%%%%%%%%%%%%%%%%%%


% Typeset the Bibliography.  The bibliography style used is "plain".
% Optionally, you can specify the bibliography style to use:
% \uofsbibliography[stylename]{yourbibfile}
\uofsbibliography{reference.bib}

% If you are not using bibtex, comment the line above and uncomment
% the line below.  
%Follow the line below with a thebibliography environmentand bibitems.  
% Note: use of bibtex is usually the preferred method.

%\uofsbibliographynobibtex


%%%%%%%%%%%%%%%%%%%%%%%%%%%%%%%%%%%%%%%%%%%%%%%%%%%%%%%%%%%%%%%%%%%%%%%%%
% APPENDICES
%
% Any chapters appearing after the \appendix command get numbered with
% capital letters starting with appendix 'A'.
% New chapters from here on will be called 'Appendix A', 'Appendix B'
% as opposed to 'Chapter 1', 'Chapter 2', etc.
%%%%%%%%%%%%%%%%%%%%%%%%%%%%%%%%%%%%%%%%%%%%%%%%%%%%%%%%%%%%%%%%%%%%%%%%%%

% Activate thesis appendix mode.
\uofsappendix

% Put appendix chapters in the appendices environment so that they appear correcty
% in the table of contents.  You can use \input's here as well.
\begin{appendices}

\chapter{Sample Appendix}

Stuff for this appendix goes here.

\end{appendices}

\end{document}
